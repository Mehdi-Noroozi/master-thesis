\chapter{Communication}
%%%%%%%%%%%%%%%%%%%%%%%%%%%%%%%%%%%%%%%%%%%%%%%%%%%%%%%%%%%%%%%%%%%%%%%%%%%%%%%%%%%%
Distributed software development (\ac{dsd}) is today practiced commonly in the software industry in both large and small companies and organizations \citep{shrivastava2010distributed}. This means that software development teams are not physically sitting in the same place and hence don’t have the opportunity to meet or speak to other team members in person regularly \citep{layman2006essential}. This distribution might mean sitting in two different buildings or sitting on two different continents. \ac{gsd} is a particular form of \ac{dsd} in which the distribution of teams are across national borderlines \citep{sahay2003global}. 

Communication has an important role in the success of \ac{gsd} \citep{carmel2001tactical,french1998study}, especially informal communication \citep{herbsleb2003empirical}. The processes of communication, coordination, and collaboration are of the utmost importance and key aspects of the software development process \citep{colomo2014agile}. 

Research shows that distance has a significant impact on the performance of projects \citep{damian2003global,Herbsleb2001a}. Communication is also an essential part of all software development practices \citep{layman2006essential}. Empirical research also shows that developers are in need of ad hoc and informal communication \citep{grinter1998recomposition,kraut1995coordination}. Therefore any shortcomings in communication between the teams and team members will profoundly impact the success of \ac{gsd} projects \citep{layman2006essential}.

Face-to-face meetings are the most efficient and ideal type of communication \citep{Kirkman2004}, but in \ac{gsd} which teams are spread in different places, often with time zone differences, this is not possible and therefore might be challenging to achieve effective communication between teams and team members. In the presence of temporal distance between the teams and team members real-time and synchronous communication using a phone, and text or video chat is challenging to achieve \citep{holmstrom2006agile,kommeren2007philips}.

\citet{casey2008impact} have worked on a case in which two distributed teams located in Ireland and Malaysia used just an asynchronous type of communication, e-mail. In that case, using asynchronous communication tools, like e-mail increased the chances of misunderstanding and ambiguity of information. Also, the communication is not adequate if the parties communicating with each other don’t understand each other \citep{kommeren2007philips,cataldo2007coordination}.

Receiving delayed feedback is also another challenge faced by remotely located teams, which itself is again a side effect of asynchronous communication \citep{conchuir2006exploring,holmstrom2006agile}. Research shows that using asynchronous communication tools in remote teams increases dramatically the time it takes to receive a response \citep{holmstrom2006global}. 

It is also true that a software engineer usually spends most of her time in searching and exchanging information, rather than doing anything else, and as a result, this can lead to delays in completing the project and increases the cost of the project. One of the ways to mitigate this problem is through proper project coordination \citep{dumitriu2006issues}. \hl{why synchronous is essential (last 2 paragraphs are about it)}

\section{Slack}
%%%%%%%%%%%%%%%%%%%%%%%%%%%%%%%%%%%%%%%%%%%%%%%%%%%%%%%%%%%%%%%%%%%%%%%%%%%%%%%%%%%%%%%%%%%%%%%%%%
Since analyzing Slack constitutes an important part of this thesis and Slack is more than yet another chat application I decided to give a description of this platform and some of its most important functions that makes Slack one of the most popular communication and collaboration systems used in software teams.

In order to describe Slack both secondary data obtained from its website, as well as my own experience as a lack user is used. The reader of this introduction will get a sense of what Slack, as a sophisticated enterprise social network is.

Slack was started in 2013 and has more than four million active users every day (Rudic, 2016). It has turned a contemporary form of communication, i.e texting, into a workplace app. Its valuation is \$ 5.1 billion. Although it has recently encountered by some big players in the market, like Microsoft, Google, and Facebook, still 43 percent of Fortune 100 companies are using Slack. Its popularity among startup companies is the same, if not more than the big enterprises (Kokalitcheva, 2016).

Williams (2015) who has created a document on how to use the Slack, says that Slack can be seen as a chat room where the whole company and its different teams can be broken into smaller channels for group discussion. Channels are often created to discuss a certain topic and are like the old chat rooms. Like any other communication platform, sending direct messages to individuals is also possible in Slack.