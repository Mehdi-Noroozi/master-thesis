\chapter{Communication in distributed software development}
%%%%%%%%%%%%%%%%%%%%%%%%%%%%%%%%%%%%%%%%%%%%%%%%%%%%%%%%%%%%%%%%%%%%%%%%%%%%%%%%%%%%
Distributed software development (\ac{dsd}) is today practiced commonly in the software industry in both large and small companies and organizations \citep{shrivastava2010distributed}. This means that software development teams are not physically sitting in the same place and hence don’t have the opportunity to meet or speak to other team members in person regularly \citep{layman2006essential}. This distribution might mean sitting in two different buildings or sitting on two different continents. \ac{gsd} is a special form of \ac{dsd} in which the distribution of teams are across national borderlines \citep{sahay2003global}. 

Communication has an important role in the success of \ac{gsd} \citep{carmel2001tactical,french1998study}, especially informal communication \citep{herbsleb2003empirical}. The processes of communication, coordination, and collaboration are of the utmost importance and key aspects of the software development process \citep{colomo2014agile}. 

Research shows that distance has a big impact on the performance of projects \citep{damian2003global,Herbsleb2001a}. Communication is also an important part of all software development practices \citep{layman2006essential}. Empirical research also shows that developers are in need of ad hoc and informal communication \citep{grinter1998recomposition,kraut1995coordination}. Therefore any shortcomings in communication between the teams and team members will heavily impact the success of \ac{gsd} projects \citep{layman2006essential}.

Face-to-face meetings are the most efficient and ideal type of communication \citep{Kirkman2004}, but in \ac{gsd} which teams are spread in different places, often with time zone differences, this is not possible and therefore might be challenging to achieve effective communication between teams and team members. In the presence of temporal distance between the teams and team members real-time and synchronous communication using a phone, and text or video chat is difficult to achieve \citep{holmstrom2006agile,kommeren2007philips}.

\citet{casey2008impact} have worked on a case in which two distributed teams located in Ireland and Malaysia used just an asynchronous type of communication, e-mail. In that case using asynchronous communication tools, like e-mail increased the chances of misunderstanding and ambiguity of information. Also, the communication is not really effective if the parties communicating with each other don’t understand each other \citep{kommeren2007philips,cataldo2007coordination}.

Receiving delayed feedback is also another challenge faced by remotely located teams, which itself is again a side effect of asynchronous communication \citep{conchuir2006exploring,holmstrom2006agile}. Research shows that using asynchronous communication tools in remote teams increases dramatically the time it takes to receive a response \citep{holmstrom2006global}. It is also true that a software engineer usually spends most of her time in searching and exchanging information, rather than doing anything else, and as a result, this can lead to delays in completing the project and increases the cost of the project. One of the ways to mitigate this problem is through proper project coordination \citep{dumitriu2006issues}. \hl{why synchronous is important (last 2 paragraphs are about it)}
