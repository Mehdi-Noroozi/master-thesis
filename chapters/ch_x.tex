\chapter{Next chapter}

Communication in teams
All software teams, either collocated or distributed interact using computer-mediated communication Kirkman  Mathieu, 2005. Basically, all teams in organizations use some IT to communicate at least some of the time, leading to the need to differentiate between teams that principally use technology as a communication medium, and teams that do so only partially.

Internal Communication 
Before the twenty-first century, communication was mainly used by organizations to create trusted, long-term relationships with external stakeholders Argenti, 2003. However, managers have more recently begun to realize that employees need the same kind of attention as they have more to do with the company’s success than any other stakeholder, resulting in a greater focus on internal communication. 
Kalla 2005, p.204 defines internal communication as “all formal and informal communication taking place internally at all levels of an organisation”. Hence, it covers both vertical up- and downward as well as horizontal between colleagues communication. Internal communication is crucial as it connects and coordinates all activities by managing the employees Holá, 2012. 
If it is done successfully it can enhance social capital King  Lee, 2016, employee engagement Karanges, Johnston, Beatson  Lings, 2015 and organizational identification Smidts, Pruyn  Van Riel, 2001. Internal communication has traditionally taken place face-to-face or in channels such as printed media, emails and phone calls Crescenzi, 2011; Smith  Mounter, 2008. However, far more alternatives are now available for organizations.

Advantages and Challenges of internet-based communication
On the one hand, teams that use internet-based communication present unique profit-enhancing advantages for organizations: reduced real estate costs; environmental benefits; improved access to global markets; and the opportunity for seamless, 24 hours per day work provided by different time zones Cascio, 2000. They also allow organizations to be more flexible and responsive to client and environmental needs because teams can tap knowledge and expertise regardless of their location Bell  Kozlowski, 2002. Early research suggests that distributed teams are actually superior to face-to-face teams for tasks such as brainstorming, due to the reduction of production blocking effects Dennis  Valacich, 1993; Gallupe, Bastianutti,  Cooper, 1991.
On the other hand, distributed teams also pose significant challenges for team members, because of the loss of social and nonverbal cues, difficulty in developing trust, as well as the demand they place upon users to be proficient in using technology Jarvenpaa  Leidner, 1999; Kayworth  Leidner, 2001. Without the nonverbal reactions to gauge feedback from the other person in the conversation, it becomes difficult to understand meta-level messages, transmit emotion, and interpret the meaning of silence Cramton, 2001. 
Consequently, misunderstandings may occur in internet-based communication, especially when users are unaware of the need to make their messages explicit. Two additional challenges apply for dispersed distributed teams: cultural due to the heterogeneity of teams and cultural misunderstandings and logistical due to various geographical zones Kayworth  Leidner, 2001. The type of technology used also impacts ease of communication, with typing being especially challenging because of lower speed and increased effort, often leading to reduced information sharing in a virtual environment Cramton  Orvis, 2003.

Media Richness and Media Synchronicity Theories 
The technology-communication relationship was captured in the seminal media richness theory Daft  Lengel, 1986, which argues that the effectiveness of transmitting information depends upon the specific medium of communication. In this view, information media span a continuum of “richness,” from letters and unaddressed documents e.g. bulk mail at the “lean,” less effective, end of the spectrum, to face-to-face, the richest medium, best capable of rendering social cues and clarifying the ambiguity.
Media synchronicity theory Dennis  Valacich, 1999 builds on the media richness theory, adding more precision by describing the characteristics of various types of media in relation to the media effectiveness for communicating information. Dennis  Valacich 1999 propose five functionalities:
    	-  Immediacy of feedback: whether the medium is synchronous or not
    	-  Symbol variety: referring to the availability of multiple cues in the medium
    	-  Parallelism: does the medium allow multiple simultaneous conversations
    	-  Rehearsability: whether communication can be rehearsed and edited prior to being transmitted 
    	-  Reprocessability: the ability of the medium to maintain a history or memory of the communication that has occurred
Various types of media score lower or higher on each of these criteria, as shown in Figure 1.
It becomes evident that communication channels, even virtual ones, are quite distinct from one another, and most likely have a differentiated impact on team interactions and outcomes; consequently, distributed team research needs to cover a multitude of media and strive towards offering in-depth explorations of each of the technologies that are currently used in organizations. 

Figure 1. Breakdown of communication media according to the criteria proposed by the media synchronicity theory. Maruping and Agarwal, 2004 
Instant Messaging
Instant messaging IM complements email and other forms of internet-based communication and offers specific characteristics and functionalities. IM is fast, quiet, and allows for synchronous collaboration. As early as 2003, the advantages of IM for businesses were being highlighted in a special news report on the technology site Cnet: “Companies are using IM not only to send real-time messages but also to collaborate on projects, exchange data and create networks linking all types of Internet devices. Employees at all levels cite many reasons that the software is well-suited for the workplace. For example, one always knows if a contact is online, at lunch or reachable on a different phone number listed on his or her status message. And IM has the immediacy of a phone call without any obligation to make small talk, saving time and therefore money.” Hu, 2003
While the corporate usage of instant messaging may have started informally, its advantages have pushed organizations into looking for viable, secure, business-tailored IM programs, which are now offered, among others, by IBM “IBM Sametime”, Microsoft “Microsoft Teams”, and Oracle.
Nowadays IM is becoming a staple of business communication worldwide: research estimates that 45.1 of employees in North American organizations currently used enterprise IM at work or for work purposes. The percentage is even higher when looking only at companies with over 500 employees, where 61.3 of employees used corporate IM systems for business purposes Osterman Research, 2010. The growth of enterprise IM is projected to outpace that of public IM globally, according to a 2013 analysis by the Radicati technology research company The Radicati Group, 2013. 
Although sometimes overlooked in the literature, the growing usage and particular manifestations of instant messaging make it a pertinent focus of analysis. IM occupies an interesting “in-between” space as a communication media: it does not replicate facial expressions such as video conferencing or vocal nuances such as audio conferencing, yet at the same time its synchronous nature prevents some of the disadvantages that plague email. It is therefore relevant to explore how teams will perform in this very particular medium.
The extant team literature that has used instant messaging confirms the premise that IM differs from face-to-face and other virtual media in terms of the effects it has on team interactions and outcomes. Acclaiming the advantages of face-to-face communication, Topi Valacich, and Rao 2002 found that face-to-face dyads required less time and were more satisfied than IM-using dyads working on a cognitive task; and in a Warren’s 2003 study, team synergy was significantly better in teams interacting face-to-face than in teams that communicated using the IM and “conference” options of FirstClass, an online collaboration tool. The face-to-face teams were also more likely to use a constructive interaction style than distributed teams. Conversely, Staples and Zhao 2006 found that, when focusing on culturally heterogeneous teams, the performance of teams communicating via IM was superior to that of face-to-face teams.
In a rare study looking at multiple types of internet-based communication, Martinez-Moreno, Gonzalez-Navarro, Zornoza,  Ripoll 2009 compared teams interacting face-to-face, via video conference, and completely computer-mediated their teams used NETMEETING, a program that has chat capabilities for communication, but it also allows users to share their desktops with one another. Researchers found that the two types of internet-based communication, video and computer-mediated, had differentiated effects on performance and were also differently impacted by the presence of conflict.

Communication Channels 
What is clear up until now is that internal communication can imply several benefits to an organization, but only if it is done in the right way. A critical task for a leader or professional communicator is to decide what tools to use in order for the communication to work. And just as the reasons for working with internal communication are many, so are the tools available for organizations to work with. Tools that all engage us in different ways and that affect the scale and pace of communication Men, 2014. According to Crescenzo 2011, these channels range from traditional channels such as printed publications, phone calls, e-mails and face-to-face communication, to Web 2.0 tools such as intranets, instant messaging and internal social networking sites. We argue that a discussion about the most common communication tools is important in order for us to be able to compare and draw conclusions about ESNs later on.

Traditional Communication Channels
One of the most common channels that organizations have traditionally used is face-to-face. Face-to-face communication requires a physical setting by both the sender and the receiver and can be divided into two different types Smith  Mounter, 2008. It can be either on a one-to-one basis through a personal meeting, or en masse when messages are to be delivered to more than one receiver. Face-to-face communication is a two-way communication channel that facilitates immediate feedback and personal focus Crescenzo, 2011. It enables a person to hear and see the non-verbal communication conveyed by the sender and respond with feedback straight away Lee, 2010, p.40. Because of these characteristics, it is often perceived as the most optimal channel for communication, especially when it comes to communicating complex information Crescenzo, 2011, new strategies and goals, or bad news Smith  Sinclair, 2003. However, it is not always accessible to all employees at all time King  Lee, 2016 and it can also be time-consuming White, Vanc  Stafford, 2010.
A two-way communication channel that offers synchronous sharing of information similar to face-to-face is phone calls Smith  Mounter, 2008. It has got some of the same advantages and disadvantages as face-to-face communication as it, for example, facilitates immediate feedback and might be time-consuming. It offers personal focus to some extent as one can hear each other. However, a big difference between the two channels is that one cannot see each other when communicating through a regular phone call.
Printed vehicles relate to paper-based communication distribution such as newspapers, magazines, and newsletters Smith  Mounter, 2008. Printed publications are perceived as a good way to ensure that important messages are stressed and elaborated, and are, therefore, often used as support material for face-to-face channels. They enable getting information out to hard-to-reach-groups, and they give time for reflection and feedback. They are also appropriate when communicating large amounts of information Sloboda, 2010. A big disadvantage of these types of channels is the cost to the environment. Furthermore, print-based channels do not facilitate a dialogue King  Lee, 2016 as it is a one-way communication tool Sloboda, 2010, implying that the receiver can ignore the information if one wishes to do so.

The other commonly used, and often viewed as traditional communication channel is email. An electronic-driven channel that, because of its very nature, is often used as a one-to-one communication tool Smith  Mounter, 2008. E-mails are often described as a fast and cheap channel of communication that allows for stretching information all over the world, allowing communicators to reach large audiences with minimum resources Sloboda, 2010; Smith  Mounter, 2008. It is also, in comparison with printed communication, environmentally friendly. Disadvantages with this kind of communication channel are that not everyone may have access to the necessary technology and that you run the risk of information overload Sloboda, 2010. Furthermore, comparing to face-to-face, e-mail is easier to misunderstand, and one may not receive an answer as fast as one would like Smith  Mounter, 2008. There is also the risk of sharing information to external and unintended audiences.

Web 2.0 Communication Tools
The advent of the Web 2.0 platforms in the last decade has changed the landscape of internal communication Smith  Mounter, 2008. Web 2.0 technology are platforms that are interactive, collaborative and participative Murugesan, 2007, and they come in several different forms such as intranets and social media. 


Intranet
An intranet is a computer network that allows people within the organizations to share information and sometimes also communicates with each other using electronic mail and different discussion forums Andersen, 2001; Smith  Mounter, 2008. A change in the use of intranets was noted by Radick 2011, who states that compared to before when intranets were mostly used to connect people to information, they are currently more and more about connecting people to people. According to Stephens, Waters, and Sinclair 2014, while workers use a wide variety of internal communication channels, as much as 85 are using intranets. 
A study conducted by Andersen 2001 showed that the use of intranets can affect organizational performance. Another study by De Bussy, Ewing, and Pitt 2004 also found that the introduction of intranets positively impacts internal communication by increasing transparency and enhancing information flows. Moreover, the advantages and disadvantages for intranets are much similar to those for e-mail use, as intranet use is cheap and environmentally friendly, but might cause information overload resulting in the receiver missing out on important information.

Social Media Platforms
Social media is another form of an Internet-driven communication tool that has evolved as a result of the Web 2.0 era. Social media can be defined as a group of Internet-based applications that build on the ideological and technological foundations of Web 2.0 that allow the creation and exchange of User Generated Content‖ Kaplan and Haenlein, 2010, p.61. Popular forms of social media include social networks, blogs, and media-sharing sites and some well-known examples are Twitter, YouTube, and Facebook. What these channels have in common is that they have two-way, interactive, personal and relational features El Ouirdi et al., 2014.
Social media allows leaders to listen to employees, respond fast, communicate in a personal way and facilitate upward communication that allows for feedback Men and Tsai, 2013, meaning it is a good channel for two-way communication King  Lee, 2016. Social media usage is found to be able to enhance internal communication in a cost- and time-efficient manner Denyer, Parry  Flowers, 2011, and to enable employees to interact with each other at any period of the day King  Lee, 2016. 
Furthermore, attributes of social media address the deficits in other channels simultaneously, such as financial and temporally limitations of face-to-face communication and the dialogue that printed media fails to provides. As mentioned earlier in this frame of reference, network externality is an important factor to consider when deciding what channels to use. And because a lot of people spend time on social media in their private life, and that social media is a platform that can be accessed irrespective of temporal and spatial limitations, network externality is expected to be high. It has been found that, through using social media within organizations, relationship promotion and recognition is improved leading to a stronger platform on which to build social capital and improve organizational performance.
Social media usage can, however, also come with some challenges. When implementing social media into the organization, you might face challenges such as lurking Ridings, Gefen  Arinze, 2006, cultural fit issues Koch, Leidner,  Gonzalez, 2013 and lack of interactivity Larsson, 2013. Smith and Mounter 2008 also mention that it is important not to get carried away with this kind of technology as not everyone will have access to it or feel comfortable using it for some time to come. Friedl and Verčič 2011, for example, found that even though young digital natives enjoy using digital media in their personal life they may not necessarily prefer to use it in their professional lives. Furthermore, it can present risks to employees‘ careers, as it can be associated with a number of risky behaviors such as wasting time or creating offensive content Landers  Callan, 2014.
A study conducted by King and Lee 2016, which focused on internal communication within the hospitality industry and included 20 semi-structured interviews with employees, showed that the employees believed that social media, as an internal communication tool, would benefit the organization. The participants argued that social media would provide the opportunity to interact with each other at any period of the day resulting in quick problem-solving. This was especially viewed as important within their industry as they work 24-hour rotational shifts and have limited opportunity to interact with their managers and co-workers. 
Another benefit that was mentioned was that it would provide them with an accessible way of reaching out to co-workers, as social media tools are available on the phone that they carry with them. The participants believed that this would make it possible to communicate asynchronously, increasing network externality as everyone can access the channel whenever they feel they have the time. As a result, they believed that social media would promote relationships and increase recognition, thereby providing a stronger platform on which to build social capital and, hence, improve organizational performance. A model, presented in Figure 2, was created to show how this process would work.

Figure 2. How social media enhances communication according to employees King  Lee, 2016. 

Enterprise Social Networks
As people have started to realize that social media can benefit organizations, a new trend has emerged. Nowadays, organizations can create private social networks restricted to employees and business partner Turban et al., 2011. As these channels are organisationally bounded and not reachable by others outside the organization, in comparison with for example Facebook and Twitter, they belong to a specific class of social media, called enterprise social networks ESN. Leonardi et al. 2013, p.19 defines enterprise social networks as web-based platforms which allow workers to: 1 communicate messages with specific coworkers or broadcast messages to everyone in the organisation; 2 explicitly indicate or implicitly reveal particular coworkers as communication partners; 3 post, edit, and sort text and files linked to themselves or others; and 4 view the messages connections, text, and files communicated, posted, edited, and sorted by anyone else in the organisation at any time of their choosing. 
Furthermore, ESNs also provide employees with a forum where they can communicate with each other publicly within the organization Leonardi, 2014. With that said, the main general functions of these tools are that one can communicate asynchronously with one person, and create group chats, in which one can send texts and documents to each other. A more in-depth explanation of different ESNs and how they function will be covered later in this section.
Even though organizations are increasingly beginning to use ESNs there is only little empirical research on the use of these platforms available today Choudrie  Zamani, 2016; Leonardi, 2014. However, we can conclude that, as ESN platforms can be seen as a type of social media, many of the advantages and disadvantages that apply to social media do also apply to ESNs. Leonardi et al 2013 do, however, argue that broad business goals, such as better access to expertise and increased knowledge sharing and innovation are better served by internal social networks. This, as they are closed resulting in higher security. Furthermore, they argue that leaders can often reach out to more employees when using an internal communication network.
The available research on ESN suggests that employees who use the tool tend to maintain connections with colleagues whom they do not know or not communicate with on a regular basis in an offline setting Leonardi, 2014. Moreover, ESNs may result in vicarious learning as one is passively exposed to what others are communicating about in open conversations. Furthermore, Leon, Rodríguez-Rodríguez, Gómez-Gasquet, and Mula 2017 argue that the use of ESNs can be very beneficial for organizations if they are properly managed. This, as it may increase employees‘ productivity and motivation, improve communication and cooperation among the organizational actors, as well as foster individual and organizational learning. Shirky 2008 states that these channels can connect groups of individuals who are not in the same physical location, and according to Qualman 2009, networked employees can be successfully involved in innovation, wealth creation and socio-economic development.
A possible problem associated with the use of ESNs is the fact that it may, in the long run, reduce employees‘ direct personal interactions which may result in psychological isolation Kane, Alavi, Labianca  Borgatti, 2014. However, Zhang and Venkatesh 2013 contradict these assumptions stating that online communication is a compliment, rather than a replacement, for offline communication. 
Moreover, Lunden 2015 states that organizations that have implemented ESNs with the goal of making their employees collaborate more may not have been successful due to the fact that employees have not been willing to learn about yet another tool. However, the probability of a successful implementation will increase if the ESN is similar to social media tools that people are using in their private life, as it will, in that case, be easier to understand. It is also argued that if an organization want to ensure a high adoption rate when implementing an ESN, it is necessary to create an open culture Korzynski, 2014. To ensure control and security, they should also introduce a code of conduct. Several ESN platforms have been developed during the last couple of years with names such as Slack, Yammer, and Workplace by Facebook Lunden, 2015.

An introduction to Slack
The following description of Slack is based on secondary data collected from Slack‘s own webpage and my own experience as I am a user of Slack. This introduction should help its audience to get a sense of what Slack is, since it is more than just a chat or instant messaging service and is more of a complex enterprise social network.
Slack is a digital workspace that helps organizations get work done by offering an easy tool for communication. It was launched to the public in 2014 and has since then grown into a popular communication tool with over four million users each day Rudic, 2016. Slack translates a modern form of communication, namely texting, into a workplace app. It has been valued at 2.8 billion, a valuation that is remarkable for its short time in business. In May 2016, 77 organizations from the Fortune 100 list were using Slack, demonstrating that it is not just an application popular among start-ups as big organizations use it too Kokalitcheva, 2016. Williams 2015 has created a guide on how to use Slack and all its features. He says that Slack is like a chat room for the whole company, where teams can be divided into smaller channels for group discussions. A channel is a room for discussions, often created for a particular topic or a specific team. Besides these group messages, people can send direct messages DM to each other, allowing for one-to-one communication.

Slack can be used on mobile phones and all kinds of desktop machines, where the user receives notifications about new messages making it easy to keep up with what is going on Williams, 2015. The user can easily manage the notifications for the different channels and messages to avoid notification overload. As with the notifications, there are many features that can be customized on Slack. One can, for example, create customized emojis and change the appearance by altering the theme and colors.
A team in Slack is a group of people that are using Slack to communicate and is likely to include the people that one is working with on an everyday basis. The team is created as a team owner simply creates a team on Slack‘s webpage. Moving forward, she invites admins who will help to organize and manage the team as they then invite the team members. If one is working in a large organization, it is likely that one is part of different teams. One can then work in different workspaces, where each workspace functions as a separate team but are all interconnected and powered via Slack‘s enterprise grid. Teams can use Slack on their computer in a browser window. It is also available as an application for iOS and Android, which means one can use it on mobile devices as well.
Below you can see a snapshot of what a typical workspace can look like, followed by a description of different functions in Slack.

Figure 3. A snapshot from Slack. 

The workplace in Slack includes channels used for group conversations. These can be organized around, for example, different departments, projects, locations, or whatever fits the organization and team members. There are two different kinds of channels; public channels and private channels. The public channels are open to the entire team. In contrast, one needs an invitation to be part of a private channel. Slack also includes a search function, used as one wishes to find documents, files, posts, messages, or team members within Slack. Messages sent in public channels are searchable by everyone in the team, while messages sent in the private channels are only searchable by the people within these channels. Moreover, teams that pay for the service can invite external parties to their workspace that will only have access to one channel within that Slack team, called single- channel guests.
Another way of communicating via Slack is through direct messages. This is used as one wish to chat with only one person DM or when one wants to start a group message Group DM with maximum eight other people. The messages sent in direct messages can only be seen and searched for by the team members in that specific direct message group. If one feels the need to talk verbally, one can make voice and video calls. One-to-one calls are available for all users, whereas group calls for up to 15 participants are available for teams who pay for the service.
As seen in figure 3, one can communicate by sending emoji. These are often used as a way to react to a message. If one wishes to get the attention from someone specific, one can type @ followed by their username. That person will then get a notification. The notifications can be managed according to one‘s own preferences as one can choose to receive notification on the desktop, mobile phone or by e-mail. Furthermore, one can customize the notifications by different channels, and set up notifications that one receives as a particular word or phrase is being communicated. One can adjust notifications to avoid receiving them at a particular time of the day, for example, if one wishes to not be disturbed during evenings and weekends. Moreover, Slack offers ways in which one can organize tasks as one can, for example, put a star or pin on messages that are of importance in order to find them more easily at a later time. One can also go back and edit a message that one has already sent. Furthermore, one can change the looks of Slack and adjust the theme and colors according to own preferences.

Slackbot is a built-in bot that helps people to get started as they join Slack. It can answer questions and can be set up to give automatic responses to team members, such as to write the Wi-Fi password if someone asks for it. Slackbot also offers a reminder function. For example, one can ask the slack bot to remind everyone in a specific channel that it is time to send information every week at a specific time.
A popular feature is to integrate Slack with different programs allowing people to save time. This can be done by for example integrating it with Google Calendar that then sends out notifications of upcoming meetings to a particular channel in Slack, or with Giphy, allowing people to quickly send GIFs to one another. A GIF consists of multiple images displayed in a succession to create an animated clip, often used as entertainment, statements or comments in online conversations William, 2016.
One can also build customized integrations suitable for one‘s team or organization Williams, 2015. Other useful features include a reminder function, where one can ask Slack to send a message to remind one of something, and an edit function, allowing one to change already sent messages, as well as a search function, which one can use to find previously sent messages. Furthermore, users can send files and folders, and even voice and video call each other on Slack Rudic, 2016. The basic features mentioned here are for free. However, companies have got the option to pay for additional features such as more storage, unlimited archived conversations, and group voice and video calls.
Advantages of using distributed teams

When assessing distributed teams and addressing the changing environment of global business it is important to know the benefits that these team structures offer a company who employs this mechanism. One of the benefits of using distributed teams is the wide range of opportunities available for resources with specific expertise or different cultural viewpoints are now available to companies. Companies are no longer restricted by a single time zone and are able to perform work continuously.

distributed teams have numerous benefits as a result of the fact that they make use of information technology to perform their tasks. According to Andriessen and Verburg 2004, some of the tools used by distributed teams to support interaction have sophisticated functionalities that provide these teams with opportunities that traditional teams do not have. One of the major effects of the introduction of collaboration technology has been that certain types of meetings can now be held with a large number of participants. Moreover, some tools allow for easy storage and retrieval of information and for collaborative editing of documents.

According to Rad and Levin 2003, one of the very attractive features of the advanced technology communication tools that are employed by a distributed team is that a team member can transfer information to any other team member. Therefore, most items of information that needs to be exchanged during the project planning and execution phases can be transmitted almost instantaneously. Since distributed teams make extensive use of information technology, they can transmit a much larger volume of information compared to the traditional information exchange modes and with greater ease.

According to Fulk and Boyd 1991, the use of advanced information technology leads to increased information accessibility, which in turn leads to changes in organization design. Both accessibility and design changes lead to improvements in the effectiveness of intelligence development and decision making. In combination, the use of computer-assisted communication technologies will lead to:

More individuals participating as information sources in decision making, but fewer persons composing the formal decision unit
Fewer organizational levels involved in processing messages and authorizing action, and a more uniform distribution across organizational levels
Greater variation across organizations in the level at which a particular decision is made
Less time devoted to decision-making meetings, and more rapid identification of problems and opportunities, action authorization, and decision making
Higher quality decisions
Fewer human links in information-processing networks.

Distributed teams can also provide greater employee flexibility. Team members typically have increased freedom in their schedules and are not necessarily confined to a traditional workday or workplace. In some regard, distributed team schedules are analogous to 'flextime' arrangements, allowing employees the ability to perform tasks on their own schedules.
 
In addition, distributed teams provide dynamic team membership and increase the number of tasks or projects that employees can work on simultaneously. It is plausible that individuals belong to more than one team at the same time and have the flexibility to move from one team to another very easily.

Moreover, organizations have the ability to quickly pool resources from a variety of locations by forming distributed teams to address specific organizational needs. In general, distributed teams allow organizations to cut travel expenses and save travel time, which results in financial savings and may increase team member satisfaction. DeRosa, Hantula, Kock and D'Arcy 2004

Disadvantages of using distributed teams

Distributed teams do have to contend with a number of disadvantages. Most of the disadvantages associated with distributed teams are related to the effectiveness of the communication between team members and in the project as a whole. As stated by Johansson, Dittrich, and Juustila 1999 the inability to cooperate and communicate effectively with other participants is one of the major barriers to the achievement of goals in projects.

The first disadvantage that distributed teams have is that since they do not communicate face-to-face many facial, verbal, social and status cues are lost in communication between team members. This may become especially important when team members are from different cultures, due to the fact that some cultures emphasize nonverbal cues and gestures in interpersonal interaction, which may lead to comprehension difficulties.

The fact that team members can be from different cities, countries, and cultures means that communication can be difficult as misunderstandings are more likely to happen. These misunderstandings can also be difficult to rectify as the team members may not be aware of these misunderstandings due to the lack of visual cues that would normally show when misunderstandings occur.

Team members might also have difficulty in feeling that they are part of a team working towards a shared goal, and might even feel socially isolated. According to DeRosa, Hantula, Kock, and D'Arcy 2004 distributed teams may also be more susceptible to problems of coordination and cohesion.

Millward and Kyriakidou 2004 mention that important socialcontextual information, such as a member's social status or level of expertise, may be lost or distorted in distributed team environments characterized by high levels of anonymity. The ability to develop relational links among team members may be hindered, which may negatively affect such outcomes as creativity, morale, decision-making quality, and process loss. Finally, the lack of social context may alter or hinder the process through which team members develop trust.

Although new and innovative modes of communication may be possible, distributed teams may still encounter significant problems in processing communication traffic among team members. According to Andriessen and Verburg 2004, other problems may include the lack of unplanned social encounters, resulting in problems with 'awareness' of availability and the state of others, of the progress of the work, or of the setting in which others work. These barriers may result in a lack of trust and cohesion, which often may lead to lower performance levels.
Introduction to coordination

We all have some intuitive sense of what coordination means, and there exist several definitions of coordination. However, currently, it does not exist a single definition that is widely accepted by everyone. One of the early definitions was proposed by Malone 1988 where he defined coordination to be ’When multiple actors pursue goals together they have to do things to organize themselves that a single actor pursuing the same goals would not have to do. We call these extra organizing activities coordination’. Some years later, Malone and Crowston 1994 redefined that definition to be ’Coordination is the managing of dependencies’. The definition is reasoned in that there are interdependent relationships between activities, and to cope with these relationships effectively, coordination mechanisms are needed Deng et al., 2007.

Osifo 2012 refers to coordination to be classified as an element in an organization. Further, he points out, as Bouckaert 2006 does, that if there is no interdependence, then there is no need for coordination either, which substantiate Malone and Crowston’s 1988 definition.
Coordination is necessary to the organization both internally and externally. Internally, because coordination is crucial to accomplishing cooperation by having participation and transparency. If there is no internal coordination, the adequate progress of the project will become difficult to achieve. Based on cooperation within the team, internal coordination also contributes to set rules and standards. External coordination is also essential by defining boundaries to establish the right vision and focus for the project Osifo, 2012.

Osifo 2012 summarizes the different aspects of a project where coordination is crucial:
’Coordination is a part of planning because it tells what to include in a good plan and how to execute it. Coordination is part of organizing because it takes the first lead. Coordination is part of staffing because it specifies who will be a staff and the rational placement. Coordination is part of directing because it gives a clear focus. Coordination is coordinating. Coordination is a part of reporting because it makes it realistic. Finally, coordination is part of budgeting, because it gives it a good appraisal’.

As the quotation reads, coordination can be found in almost every part of a project, and should, therefore, be paid attention to be able to optimize the work. Coordination is seldom exercised alone through a single coordination mechanism. It is most often achieved through several mechanisms that all together achieve the overall coordination for a project or an organization Dietrich et al., 2013.

Coordination in large-scale projects

Software development in larger scale have in the later years met challenges, especially in relation to coordination between teams. As projects increase in size and complexity, the need for coordination also increases Kraut and Streeter, 1995. There has been an increasing use of team’s setup in projects because of increasing complexity, and the organizations thereby need different approaches to handling these changes and challenges with coordination Scheerer et al., 2014. When dealing with large groups of people that need to be coordinated, this often ends up in a hierarchical team of team’s setup, which in organizational theory is defined as a multi-team system MTS.

Scheerer et al., 2014. Mathieu et al.2001 define MTS as:
’Two or more teams that interface directly and interdependently in response to environmental contingencies toward the accomplishment of collective goals. MTS boundaries are defined by virtue of the fact that all teams within the system while pursuing different proximal goals, share at least one common distal goal; and in doing so exhibit input, process, and outcome interdependence with at least one other team in the system’.

In software development, a common way of handling inter-team coordination, in a similar environment as MTS, is according to research done by Scheerer et al. 2014, the Scrum-of-Scrums approach, which is substantiated by other researchers as well Larman, 2008, Schwaber, 2004.

When it comes to large scale software development projects, an increasing effort must be put into coordination to get all the work done and being able to work together without too much redundancy Kraut and Streeter, 1995. In the process of developing software, there is a need for tight coordination among the involved to produce a successful system. Coordination is crucial in large projects, however it can be difficult to achieve.

Agile software development in a large-scale deal with multiple teams that require some sort of inter-team coordination, because of increased complexity. When teams grow in size, the number of inter-team dependencies also tend to increase. More coordination effort is then needed to deal with inter-team dependencies so that each team’s individual goal is reached, and also the overall goal of the project is achieved Larman and Vodde, 2010, Paasivaara et al., 2012, Scheerer et al., 2014. Bick et al. 2016 have conducted a research where they studied five different ways of practicing agile at scale. What they experienced in that study was that inter-team coordination approaches vary a lot when it comes to their nature of coordination Bick et al., 2016.


Curtis et al.1988 present in their study that coordination in large projects can be challenging, as there are more likely to be communication bottlenecks and breakdowns. Large-scale projects also come with more uncertainties that also affect coordination, as specifications of the requirements might change over time. This can be unpredictabilities when it comes to both the software, and the tasks the team members shall perform Kraut and Streeter, 1995. There can also be uncertainties within the teams that affect coordination. For instance, what should be prioritized first and how to do tasks, as different people might have different opinions? If coordination between the various teams, and within the team, is weak, this alone can contribute to integration failure.

Marks et al. 2005 describe the boundary of an MTS based on Mathieu 2001 as when ’teams share input, process, and outcome interdependence with at least one other team in the MTS network’. This has similarities to the case for agile teams in large-scale development projects, where the teams are dependent on the other teams to be able to deliver a product successfully. This substantiates the need for some kind of inter-team coordination. Another study that has pointed out inter-team coordination to be crucial is Melo et al. 2013. They identify agile team management to be the most influential factor in the productivity of agile teams. In relation to this, their case study showed that inter-team coordination emerged as an important inter-team management issue Melo et al., 2013. Inter-team coordination influence team management productivity. If there, for instance, is a lack of commitment by one team, which can end up in delays and misalignment, or if there are too strict rules of coordination among the teams, this will lead to less agility.

Van de Ven model of coordination

In organizational theory, Mintzberg 1980 proposes, based on earlier research, that mutual adjustment, direct supervision, and standardization to be mechanisms for handling coordination. Strode et al. 2012 present valuable insight to coordination mechanisms related to agile development in small-scale. They present a coordination strategy that includes the mechanisms synchronization, structure, and boundary spanning to be valuable for effective coordination. Their study also points out that from an agile perspective, coordination is often achieved through mutual adjustment at a group level. At the individual level, coordination is obtained by personal horizontal coordination by one-to-one communication Strode et al., 2012.

Thompson 1967 divide coordination into three based on the type of inter-dependencies: pooled, sequential and reciprocal. Pooled are when units within an organization accomplish completely separate tasks and do not interact implicit dependency to other entities. Standardisation best coordinate this with little communication and decision effort. Sequential arises when one unit depends on the output of another to continue its work. This is best coordinated by planning and medium effort for communication and decision effort. Reciprocal occurs when the input and output flow in both directions simultaneously between the dependent units and is coordinated by feedback and mutual adjustments.

Van de Ven et al. 1976 is another coordination model that is to some extent similar to the findings of the other theorists but adds the dimension of a team to their coordination strategy. This means, for instance, that mutual adjustment is extended by collective interactions within teams who are usually co-located. However, it has many similarities to how Thompson 1967 suggests in theory that coordination can be achieved. The Van de Ven model of coordination is, because of the team aspect, interesting concerning inter-team coordination, which has been mentioned to be an essential factor to coordination in large-scale projects.

According to research conducted by Van de Ven et al. 1976, there are three modes for coordinating work activities; impersonal, personal, and group mode of coordination which has several similarities to what is suggested by Thompson 1967. These will be presented in more depth later.

In the study Van de Ven et al. 1976 conducted, they wanted to examine to which extent the factors task uncertainty, task interdependence, and unit size could predict variations in the use of the three modes of coordination.
Van de Ven et al. 1976 state that there are some fundamental factors which explain why there are different mechanisms for coordination within an organization; namely, task uncertainty, task interdependence, and unit size. They state that in different situations, it is possible to determine when one, or a combination of different coordination mechanisms, is used, dependent on these three factors.

Task uncertainty: is considered the difficulty and variability of the work. This can, for instance, be measured in how analyzable the work is, and if the work methods are predictable Van de Ven et al., 1976. Van de Ven et al. also list other measures of task uncertainty; 1 the degree of complexity of the search processes; 2 the amount of thinking time to solve problems; 3 the extent to which task processes or interventions have knowable outcomes; 4 the amount of time required before outcomes are known.
Task interdependence: concerns to what degree a task is dependent upon one another, and to what extent it is possible to do individual jobs separately.
Unit size: is in this context related to the total number of people employed in a work unit.

These three factors, which Van de Ven et al. 1976 mean are fundamental, are used to explain the usage of different coordination modes. The various modes are needed to be able to increase benefits of the project, as well as deal with potential challenges.

In the study conducted by Van de Ven 1976, it is also stated some interesting relations regarding what happens as unit size increases:
1. Group cohesiveness decreases and sub-group formation increases.
2. Member participation decreases and more mechanical methods are used to introduce information, and a more direct attempt is made to control the behaviors of participants in reaching a solution.
3. Face-to-face techniques of leadership behavior give away to more impersonal techniques of coordination.
4. Demands on the leaders become more complex and numerous, and group members become more tolerant of highly structured and directive leadership.

Van de Ven’s model of coordination has a high focus on the team aspect and how coordination modes change, depending on influential factors which may affect coordination mechanisms. Coordination of large-scale projects is complex, and from what can be learned from the Van de Ven model is that coordination is a changing mechanism, and it is rarely exercised through only one coordination mechanism. Several aspects influence coordination, and as projects become larger, it becomes more challenging to deal with it appropriately.

Here I will present the different modes of coordination: impersonal, personal, and group mode, and substantiate them with theory from other studies to see their relation to inter-team coordination. Zmud 1980 describes these modes to be three predominant coordination modes that enable information processing. Espinosa 2004 suggest in their study that different modes of coordination are needed as the coordination modes are suitable for various tasks. They also point out the fact that the same task might require different coordination modes over time Espinosa et al., 2004. This substantiates Van de Ven’s model of dividing coordination into coordination modes suited for different purposes. Van de Ven et al. 1976 suggest that coordination mechanisms within each of the different modes are used in various combinations to achieve a collective goal.

Impersonal mode of coordination

Impersonal mode of coordination relates to anything that has to do with programming, administrative coordination, and technical tools, and once it is implemented, its use requires minimal verbal communication between actors Boos et al., 2011, Van de Ven et al., 1976. These principles are also suggested in theory by Kraut et al. 1995, that point out the combination of large size, uncertainty, and interdependence require particular coordination techniques like technical tools, modularization, and formal procedures. These techniques do not remove all challenges of coordination but can help to ease some of them.

Impersonal mode of coordination helps to ease of coordination issues within a large project and is often seen as increased in importance as a project gets more complex. Mintzberg 1980 conducted one of the earlier studies on coordination and suggested several coordination mechanisms that could contribute to how organizations could coordinate their work more efficiently. One of these mechanisms, namely standardization, substantiate the impersonal mode of coordination that Van de Ven et al. 1976 have identified. According to Mintzberg 1980 standardization involve three aspects:

Standardisation of skills: is often the case in the initial phase of a project for instance, where coordination is achieved through standardization of skills and knowledge through training and education.
Standardisation of work processes: where one is using standards, e.g. rules, routines or regulations, to guide how to perform a certain activity.
Standardisation of outputs: gives coordination by communicating and clarifying what is expected of the results.

Another study which also has similarities to an impersonal mode of coordination is one of the classifications of Espinosa’s coordination mechanisms Espinosa et al., 2010. Espinosa 2010 define mechanistic coordination to concern coordination by a program or by a plan with the use of artifacts, processes, and routines to deal with dependencies with little communication. The definition of mechanistic coordination is one of three mechanisms Espinosa 2010 identify to deal with coordination, and as Van de Ven et al. 1976, Espinosa state that there is a need for several mechanisms to deal with coordination.

The use of tools to organize and have a common platform to share reports, guidelines, etc. are important in large-scale projects to keep coordinated. Malone and Crowston 1994 contributed, with their early work on coordination, with a theoretical modeling framework to be used for analyzing complex coordination processes. They saw the benefit of using this framework to examine group action, regarding actors performing interdependent tasks, which either create or require different resources Crowston et al., 2004. Tasks can be seen as system requirements that are translated from the requirements of the customer. While resources contain information like problem description, existing system functionality, and time and cost analyses Crowston et al., 2004. However, the theory has shown in more recent times that it is not that suitable to predict for instance coordination effectiveness and that not all coordination mechanisms can be seen as general. That said, it is still a valuable framework to gain a better understanding of how the different factors support coordination Strode et al., 2012. This also substantiates that some frameworks, guidelines, plans - some artifacts to keep organized and coordinated - are necessary, as with impersonal mode of coordination.

Personal mode of coordination

With the personal mode of coordination, Van de Ven et al. 1976 identify that coordination occurs as feedback by mutual adjustments regarding the input information one receives. The individual makes mutual task adjustments either through vertical or horizontal communication channels. Vertical communication is usually line managers and unit supervisors, while horizontal communication is concerned about individuals in teams having one-to-one communication with someone. With horizontal communication, there is often a non-hierarchical relationship between the actors.

As suggested by Mintzberg 1980, there are two coordination mechanisms; mutual adjustment and direct supervision which concerns the same principles as with vertical and horizontal communication in a personal mode of coordination.

Mutual adjustment: when team members are using informal communication with others to coordinate their interdependent work.
Direct supervision: where there typically is one individual, e.g. the manager, that gives specific orders to the rest of the team, and the team thereby coordinate and take responsibility for that work is being conducted.

These two mechanisms substantiate the need for the personal mode of coordination, and that through mutual adjustments within a team, coordination can be achieved. Also, Espinosa’s 2010 classification of organic coordination has some similar aspects as Mintzberg in relation to the personal mode of coordination. Organic coordination is related to coordination that is achieved by feedback or by mutual adjustment, and thereby the coordination is mainly accomplished by communication and interaction. The communication can be informal and spontaneous or formal and planned Espinosa et al., 2010.

Personal mode of coordination includes both formal and informal communication which have been identified to be important to coordination Kraut and Streeter, 1995. Especially, when dealing with a high degree of uncertainty, Kraut et al. 1995 state that the informal, interpersonal communication is valuable both for team members and for the project as a whole. Formal and informal communication bring value to a project, as coordination increases when sharing information. Another important factor for coordination in software development, as Kraut et al. 1995 point out, is that the personal communication that finds a place across functional boundaries that help to handle uncertainty. This substantiates that personal mode of coordination is a critical mode to deal with inter-team coordination to ease uncertainty.

Theory by Boos et al. 2011 describe personal mode as a useful coordination mode when things are not scheduled and anticipated, and they put communication between the team members, as a dependency factor for the personal mode of coordination. This can also be found in theory by Dickinson and McIntyre 1997 which identify communication as "the glue" of teamwork since it links all the other components. Boos et al. 2011 identify several measurement levels for whether personal mode will succeed, such as planning, information exchange, feedback; by giving, seeking and receiving information between teams, and leadership Boos et al., 2011. As suggested by Kraut et al. 1995, coordination via communication does not only occur formally via meetings and documents, on the contrary, a substantial amount of coordination happens with informal communication for instance in cafeterias or hallways. This substantiates the advantages of personal mode of coordination

As a personal mode of coordination mainly concerns communication, both horizontally and vertically, some factors challenge this coordination mode which can be crucial to whether the coordination is successful or not. Lehtimäki 1996 state the importance trust has with coordination. This is caused by coordination creates the network where organizational performance is understood. Trust is then vital when project increase in complexity, but may also be harder to accomplish them. Better performance can be achieved by having good coordination, and then the network of trust is essential for coordination Osifo, 2012. The lack of trust has been identified as a central factor to poor coordination and cooperation Smith and Schwegler, Smith and Schwegler. Osifo 2012 concludes, based on various coordination literature, that it is visible that trust is a part of performance since it creates a foundation for proper coordination.

Group mode of coordination

With group mode, coordination also occurs by feedback as mutual adjustments as with personal mode of coordination; however, the mutual adjustments occur through the group by scheduled or unscheduled staff or committee meetings Van de Ven et al. 1976. It involves new routine and is usually more planned communication.

Group mode of coordination is an important mode, especially where several teams need to be coordinated. Espinosa’s 2010 classification of cognitive coordination refers to the knowledge the actors have about each other, as well as the tasks the others are doing. This coordination can be beneficial in relation to knowing what others are likely to do and with group mode one can achieve coordination by having meetings evenly to keep updated on each other. Espinosa et al. 2010 state that cognitive coordination can be seen as a critical enhancer of mechanistic and organic coordination which is important for the impersonal and personal mode of coordination.

Other researchers have also identified aspects of group mode of coordination to be important to deal with coordination in large projects. Dietrich et al. 2013 identified three specific coordination mechanisms through their study of multi-team projects which are centralized, decentralized, and balanced patterns. Centralised coordination concerns coordination that occurs at the group level, like scheduled and unscheduled meetings to make adjustments. Decentralised coordination concerns coordination that occurs between team members, which not necessarily are pre-defined meetings. Balanced coordination involves a combination of the two previous.

The results of their study show that the diversity in coordination practices have an influence on several aspects of the project like ’information sharing, workflow fluency between teams, the efficiency of the project, and learning outcomes’ Dietrich et al., 2013. Dietrich’s study substantiates the importance of having different kinds of coordination mechanisms with multi-team systems Mathieu et al., 2001 to handle inter-team coordination, and that several levels of coordination are important to consider when many teams are involved. Thereby, group mode of coordination is considered essential in large-scale projects.

In large-scale projects, one can find several kinds of dependencies that urge the need for coordination. With the proper coordination, it enables collaboration among the different teams Melo et al., 2013. Dependencies need to be dealt with and compatible with the needs of the teams. Malone and Crowston 1994 conducted an interdisciplinary study of coordination where the key insight was that coordination could be seen as the process of managing dependencies among activities. These activities can be seen as constraints on an action in a situation Malone and Crowston, 1994, Strode et al., 2012. A result of their study is their definition of coordination that is ’managing dependencies between activities’. Their theory is built on ideas from the fields of organization theory, economics, management, and computer science, and can be thereby being seen as an interdisciplinary study of coordination. Group mode of coordination is one way of dealing with this as dependencies will always occur, and these need to be addressed also across the teams. That mode of coordination is suited to be able those dependencies across the teams and is thereby necessary for large-scale agile.

Strode et al. 2012 have developed a coordination strategy that includes several mechanisms to cope with dependencies in a situation. By identifying several specific strategies for unique cases, the formation of general coordination strategy concept was developed Strode et al., 2012. The coordination strategy concept then resulted in three main components: synchronization, structure, and boundary spanning which are mechanisms to help manage dependencies. The synchronization is of particular importance concerning group mode of coordination. It is achieved through synchronization activities and artifacts that are produced during those activities. These activities are meant to bring the whole team together at the same time and place for a pre-arranged purpose. Many of these activities often occur only once, for instance at the beginning of a project to agree upon technical decisions, to develop a high-level project scope, and to define the initial requirements. However, there are also synchronization activities during the project. The purpose of these activities is to gain a common understanding which is really important to the teams and thereby important for group mode.

The coordination that occurs with group mode is often directly related to inter-team coordination and issues that are necessary to deal with across the teams. This mode is important to have a balanced coordination Dietrich et al., 2013 to increase information sharing, efficiency, workflow fluency and overall result of a project.

