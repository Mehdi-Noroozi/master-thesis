\chapter{Introduction}
%%%%%%%%%%%%%%%%%%%%%%%%%%%%%%%%%%%%%%%%%%%%%%%%%%%%%%%%%%%%%%%%%%%%%%%%%%%%%%%%%%%%%%
This chapter describes the motivation for the topic of this thesis, states its research questions and sums up the thesis structure. The motivation for the study is based on the globalization of software development, which in turn demands better communication and coordination practices for the distributed software development teams. This type of communication and coordination in distributed software projects involves severe and persistent challenges. This study, therefore, investigates the communication and coordination challenges faced by distributed software projects.


\section{Motivation and background}
%%%%%%%%%%%%%%%%%%%%%%%%%%%%%%%%%%%%%%%%%%%%%%%%%%%%%%%%%%%%%%%%%%%%%%%%%%%%%%%%%%%%%%
Globalization has become a fact in recent years, and outsourcing of resources to gain access to cheaper work-force or expert knowledge has never been higher than today in the history \citep{MadonShirinandKrishna2017}. Statistics show that this trend is just continuing and that the software industry, because of its unique characteristics will be affected even more than other industries \citep{Sarfraz2016}. 

One of the features that makes global software development (\ac{gsd}) so interesting and available for outsourcing is the possibility of working on the same software project simultaneously from wherever in the world, given having access to the Internet or another type of network which can connect all the people involved in the project connected to each other \citep{Ferguson2004}. However, at the same time, while this scenario is so attractive on paper, when it comes to practice there are many obstacles that a distributed software team (\ac{dst}) might face. 

Communication and coordination between team members and other stakeholders turn to become a big challenge in \ac{dst} \citep{Casey2005}. The author was personally involved in a \ac{dst} before and seeing all the challenges faced by the team became motivated to conduct a study in this area to identify the challenges in communication and coordination between \ac{dst} members and possibly come with some suggestions to improve those practices.


\section{Research objective}
%%%%%%%%%%%%%%%%%%%%%%%%%%%%%%%%%%%%%%%%%%%%%%%%%%%%%%%%%%%%%%%%%%%%%%%%%%%%%%%%%%%%%%
In the absence of a thorough understanding of communication and coordination challenges in large-scale distributed software development \ac{dsd} projects, this master thesis will try to contribute to increasing the awareness of such challenges. Thus, this master thesis aims to answer the following research questions:

\begin{quote}
\textbf{Research question 1:} What are the challenges of communicating and coordinating in not co-located teams, and what kind of consequences might it have for distributed software teams?\\
\textbf{Research question 2:} How are communication and coordination is perceived by the distributed team members?
\end{quote}

The motive for these questions is to look more into the topic of communication and coordination in distributed software development by using both quantitative and qualitative methods. In this regard I decided to analyze a case and run a survey to achieve a deeper understanding of the topic. Through conducting research with the goals mentioned above in mind, it should be possible to spot practices used to achieve better coordination and communication in distributed software teams, as well as challenges that could be identified through analyzing a real case. 


\section{Thesis Structure}
%%%%%%%%%%%%%%%%%%%%%%%%%%%%%%%%%%%%%%%%%%%%%%%%%%%%%%%%%%%%%%%%%%%%%%%%%%%%%%%%%%%%%%
\hl{Fill this place later}



