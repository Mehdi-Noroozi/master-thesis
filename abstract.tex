\begin{abstract}
Most students regard the abstract as one of the last things - along with acknowledgements, title page and the like - that they are going to write. Indeed, the final version of the abstract will need to be written after you have finished reading your thesis for the last time.

However, if you think about what it has to contain, you realise that the abstract is really a mini thesis. Both have to answer the following specific questions:      
\begin{enumerate}
\item What was done?
\item Why was it done? 
\item How was it done?
\item What was found?
\item What is the significance of the findings?
\end{enumerate}

Therefore, an abstract written at different stages of your work will help you to carry a short version of your thesis in your head. This will focus your thinking on what it is you are really doing , help you to see the relevance of what you are currently working on within the bigger picture, and help to keep the links which will eventually unify your thesis.

\textbf{Process}\\
The actual process of writing an abstract will force you to justify and clearly state your aims, to show how your methodology fits the aims, to highlight the major findings and to determine the significance of what you have done. The beauty of it is that you can talk about this in very short paragraphs and see if the whole works. But when you do all of these things in separate chapters you can easily lose the thread or not make it explicit enough.

If you have trouble writing an abstract at these different stages, then this could show that the parts with which you are having a problem are not well conceptualized yet.
We often hear that writing an abstract can't be done until the results are known and analyzed. But the point we are stressing is that it is a working tool that will help to get you there.

Before you know what you've found, you have to have some expectation of what you are going to find as this expectation is part of what is leading you to investigate the problem. In writing your abstract at different stages, any part you haven't done you could word as a prediction. For example, at one stage you could write, "The analysis is expected to show that …". Then, at the next stage, you would be able to write "The analysis showed that …." or "Contrary to expectation, the analysis showed that …..".

The final, finished abstract has to be as good as you can make it. It is the first thing your reader will turn to and therefore controls what the first impression of your work will be. The abstract has to be short-no more than about 700 words; to say what was done and why, how it was done, the major things that were found, and what is the significance of the findings (remembering that the thesis could have contributed to methodology and theory as well). 

In short, the abstract has to be able to stand alone and be understood separately from the thesis itself. 
\end{abstract}