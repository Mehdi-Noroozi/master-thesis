\begin{abstract}


Globalization is affecting our daily life more and more every day. In addition to affecting every and each person, it is also affecting businesses around the globe and to cope with it managers and other decision-makers should think of new and innovative ways to keep their businesses going on. Software development industry is also no exception and was one of the industries that since the 1990s started to merge in the global market. The expansion of access to the internet around the world has evaded the need for co-located software teams and has made it possible to freely collaborate and work on software projects from all around the world. \\

There are several success stories regarding the distributed and global software development, however, it is not free from mistakes and huge financial losses. Among the several issues that distributed software development is facing, communication and coordination are named in several studies as the prominent cause of failure in distributed projects. In the absence of face-to-face communication which is considered to be the richest form of communication and the presence of cultural and language barriers, misunderstanding is unavoidable and problems can easily escalate.\\

In this thesis, we are going to take a closer look at communication and coordination issues in distributed software teams through analyzing Slack chat logs of two distributed teams as well as running a survey among software developers. In the end, we are going to give some suggestions for improvement of communication and coordination in distributed teams.\\

The results gained through the analysis of Slack chat logs and results of the survey are in agreement with several other studies and research done on the topic. 


\end{abstract}