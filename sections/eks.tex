\chapter{Example}


Start off all chapters with \verb|chapter|. \index{chapter!numbered} \verb|\extrachapter| will give you an unnumbered chapter that's added to the Table of Contents. \index{chapter!unnumbered}


If you're new to \LaTeX{} and would like to begin by learning the basics, please see our free online course available at:\\ \url{https://www.overleaf.com/latex/learn/free-online-introduction-to-latex-part-1} \index{LaTeX@\LaTeX}


\section{This is a Section}

\begin{figure}[hbt!]
\centering
\includegraphics[width=.3\textwidth]{uio.png}
\caption{This is a figure}\label{fig:logo}
\index{figures}
\end{figure}

\subsection{This is a subsection}

\begin{table}[hbt!]
\centering
\begin{tabular}{ll}
\hline
Area & Count\\
\hline
North & 100\\
South & 200\\
East & 80\\
West & 140\\
\hline
\end{tabular}
\caption{This is a table}
\label{tab:sample}
\index{tables}
\end{table}


Here's an endnote.\endnote{Endnotes are notes that you can use to explain text in a document.}

\section{This is code Section}

\begin{lstlisting}[language=Python, caption=JSON example]
 {
     "type": "message",
     "user": "U3MPT6W77",
     "text": "I don't understand the question :slightly_smiling_face:",
     "ts": "1495181765.592074"
  },
\end{lstlisting}
